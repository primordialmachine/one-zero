%%%%%%%%%%%%%%%%%%%%%%%%%%%%%%%%%%%%%%%%%%%%%%%%%%%%%%%%%%%%%%%%%%%%%%%%%%%%%%%%%%%%%%%%%%%%%%%%%%%
%
% Primordial Machine's One Zero Functors Library
% Copyright (C) 2017-2019 Michael Heilmann
%
% This software is provided 'as-is', without any express or implied warranty.
% In no event will the authors be held liable for any damages arising from the
% use of this software.
%
% Permission is granted to anyone to use this software for any purpose,
% including commercial applications, and to alter it and redistribute it
% freely, subject to the following restrictions:
%
% 1. The origin of this software must not be misrepresented;
%    you must not claim that you wrote the original software.
%    If you use this software in a product, an acknowledgment
%    in the product documentation would be appreciated but is not required.
%
% 2. Altered source versions must be plainly marked as such,
%    and must not be misrepresented as being the original software.
%
% 3. This notice may not be removed or altered from any source distribution.
%
%%%%%%%%%%%%%%%%%%%%%%%%%%%%%%%%%%%%%%%%%%%%%%%%%%%%%%%%%%%%%%%%%%%%%%%%%%%%%%%%%%%%%%%%%%%%%%%%%%%

\documentclass[oneside]{book}

% Copyright (c) 2018 Michael Heilmann. All rights reserved.

\makeatletter

% "(Get|Set)Author".
% The name of the author.
\def\SetAuthor#1{\gdef\@author{#1}}
\def\@author{\@latex@warning@no@line{No author given}}
\def\GetAuthor{\@author}

% "(Get|Set)Email".
% The email (address) of the author.
\def\SetEmail#1{\gdef\@email{#1}}
\def\@email{\@latex@warning@no@line{No email given}}
\def\GetEmail{\@email}

% "(Get|Set)Organization".
% The organization the library is published by.
\def\SetOrganization#1{\gdef\@organization{#1}}
\def\@organization{\@latex@warning@no@line{No organization given}}
\def\GetOrganization{\@organization}

% "(Get|Set)LibraryName".
% The name of the library.
\def\SetLibraryName#1{\gdef\@libraryName{#1}}
\def\@libraryName{\@latex@warning@no@line{No library name given}}
\def\GetLibraryName{\@libraryName}

% "(Get|Set)LibraryVersion".
% The version of the library.
\def\SetLibraryVersion#1{\gdef\@libraryVersion{#1}}
\def\@libraryVersion{\@latex@warning@no@line{No library version given}}
\def\GetLibraryVersion{\@libraryVersion}

% "(Get|Set)LibraryRepository".
% The repository (url) of the library.
\def\SetLibraryRepository#1{\gdef\@libraryRepository{#1}}
\def\@libraryrepository{\@latex@warning@no@line{No library repository given}}
\def\GetLibraryRepository{\@libraryRepository}

% Define "maketitle".
\def\maketitle{%
  \noindent\makebox[\textwidth]{%
	\uppercase{{\GetOrganization} {\GetLibraryName} {\GetLibraryVersion}}%
	\hfill
	\uppercase{{\GetAuthor} {(\href{mailto:\GetEmail}{\GetEmail})}}%
  }%
}

\makeatother


\SetOrganization{Primordial Machine's}
\SetLibraryName{One Zero Functors Library}

\SetLibraryIncludeFileName{include.hpp}
\SetLibraryIncludesDirectoryPath{primordialmachine/one-zero-functors/\newline\$(PlatformTarget.toLower())/\$(Configuration.toLower())/includes}

\SetLibraryIncludeDirectiveFilePath{primordialmachine/one\_zero\_functors/include.hpp}

\SetLibraryStaticLibrariesDirectoryPath{primordialmachine/one-zero-functors/\newline\$(PlatformTarget.toLower())/\$(Configuration.toLower())/libraries}
\SetLibraryStaticLibraryFileName{one-zero-functors.lib}

\SetLibraryVersion{v1.5}
\SetLibraryRepository{https://github.com/primordialmachine/one-zero-functors}
\SetAuthor{Michael Heilmann}
\SetEmail{michaelheilmann@primordialmachine.com}

\SetDocumentType{User Manual}

\begin{document}

\frontmatter

\begin{titlepage}
\maketitle
\end{titlepage}

\tableofcontents
\addtocontents{toc}{\protect\thispagestyle{empty}}
\pagenumbering{gobble}

\mainmatter

\chapter{Synopsis}
C++ 17 library providing extendable abstractions of the constants zero and one.\newline

\noindent{}This library provides functors which return the constant zero and one for most built-in types.
Furthermore, a consumer of this library can add additional partial specializations of these functors for arbitrary types.\newline

\noindent{}The library is made available publicly on
\href{\GetLibraryRepository}{Github}
under the
\href{\GetLibraryRepository/blob/master/LICENSE}{MIT License}.

\chapter{Requirements}
None.

\chapter{Limitations and Restrictions}
The library officially only supports Visual Studio 2017 and Windows 10.

\chapter{Introductory example}
\textit{\color{orange}This library does not provide any examples yet.}
%Examples are located in the \href{\GetLibraryRepository/blob/master/examples}{examples} directory.

\input{building_visual_studio_2017}

\chapter{Library Interface Documentation}

\section{\texttt{namespace primordialmachine}}
The namespace this library is adding its declarations/definitions to.
The added namespace elements are documented below.

\section{\texttt{zero\_functor\textlangle TYPE\textrangle}}
\begin{verbatim}
template<typename TYPE>
struct zero_functor;
\end{verbatim}
Provides a const constexpr noexcept \texttt{operator()} which receives zero arguments and returns a
value of type \texttt{TYPE} approximating the value $0$ (zero).
Specializations for the types
\texttt{char},
\texttt{signed char},
\texttt{unsigned char},
\texttt{signed short int}       (aka \texttt{short int},     aka \texttt{signed short},     aka \texttt{short}),
\texttt{signed long int}        (aka \texttt{long int},      aka \texttt{signed long},      aka \texttt{long}),
\texttt{signed long long int}   (aka \texttt{long long int}, aka \texttt{signed long long}, aka \texttt{long long}),
\texttt{unsigned short int}     (aka \texttt{unsigned short}),
\texttt{unsigned long int}      (aka \texttt{unsigned long}), and
\texttt{unsigned long long int} (aka \texttt{unsigned long long}),
\texttt{float},
\texttt{double}, and
\texttt{long double}
are provided.

\section{\texttt{one\_functor\textlangle TYPE\textrangle}}
\begin{verbatim}
template<typename TYPE>
struct one_functor;
\end{verbatim}
Provides a const constexpr noexcept \texttt{operator()} which receives zero arguments and returns a
value of type \texttt{TYPE} approximating the value $1$ (one).
Specializations for the types
\texttt{char},
\texttt{signed char},
\texttt{unsigned char},
\texttt{signed short int}       (aka \texttt{short int},     aka \texttt{signed short},     aka \texttt{short}),
\texttt{signed long int}        (aka \texttt{long int},      aka \texttt{signed long},      aka \texttt{long}),
\texttt{signed long long int}   (aka \texttt{long long int}, aka \texttt{signed long long}, aka \texttt{long long}),
\texttt{unsigned short int}     (aka \texttt{unsigned short}),
\texttt{unsigned long int}      (aka \texttt{unsigned long}), and
\texttt{unsigned long long int} (aka \texttt{unsigned long long}),
\texttt{float},
\texttt{double}, and
\texttt{long double}
are provided.

\section{\texttt{zero\textlangle TYPE\textrangle()}}
Function returning an approximation of type \texttt{TYPE} of the value $0$ (zero).
\begin{verbatim}
template<typename TYPE>
constexpr auto zero() -> decltype(zero_functor<TYPE>()())
{ return zero_functor<TYPE>()(); }
\end{verbatim}

\noindent{}Provides a constexpr \texttt{operator()} which receives zero arguments and returns a value of type
\texttt{decltype\allowbreak(zero\allowbreak\_functor\allowbreak<TYPE>\allowbreak()())} approximating the value $0$ (zero).
The function is noexcept if \texttt{zero\allowbreak\_\allowbreak functor<TYPE>()()} is noexcept otherwise not.

\section{\texttt{one\textlangle TYPE\textrangle()}}
Function returning an approximation of type \texttt{TYPE} of the value $1$ (one).
\begin{verbatim}
template<typename TYPE>
constexpr auto one() -> decltype(one_functor<TYPE>()())
{ return one_functor<TYPE>()(); }
\end{verbatim}

\noindent{}Provides a constexpr \texttt{operator()} which receives zero arguments and returns a value of type
\texttt{decltype(one\allowbreak\_functor\allowbreak<TYPE>\allowbreak()())} approximating the value $1$ (one).
The function is noexcept if \texttt{one\allowbreak\_\allowbreak functor<TYPE>()()} is noexcept otherwise not.

\end{document}
